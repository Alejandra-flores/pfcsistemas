\section*{Agradecimientos}

Este trabajo de fin de carrera significa la finalización de mis estudios de primer ciclo, por lo que me gustaría agradecérselo a todas las personas que en menor o mayor medida han hecho posible que me encuentre hoy en esta situación.\\

En primer lugar me gustaría agradecerle a mi familia su ayuda y apoyo durante estos largos años, y el tesón que han demostrado hacia mí en los momentos más difíciles.\\

También quiero mencionar a mis compañeros de carrera, sin su inestimable ayuda no habría conseguido lograr mi meta con una sonrisa.\\

Por último quiero dedicarle este proyecto a todos los interesados en la informática teórica, en concreto en la teoría de grafos y las matemáticas aplicadas a la informática que tantas horas de entretenimiento pueden darte.\\


\cleardoublepage

\section*{Licencia}

Este documento ha sido liberado bajo Licencia GFDL 1.3 (GNU Free
Documentation License). Se incluyen los términos de la licencia en
inglés al final del mismo.\\

Copyright (c) Moisés Gautier Gómez.\\

Permission is granted to copy, distribute and/or modify this document
under the terms of the GNU Free Documentation License, Version 1.3 or
any later version published by the Free Software Foundation; with no
Invariant Sections, no Front-Cover Texts, and no Back-Cover Texts. A
copy of the license is included in the section entitled "GNU Free
Documentation License".\\

\cleardoublepage

\section*{Notación y formato}

Definiremos un grafo como un sistema matemático abstracto. No obstante, para desarrollar el conocimiento de los mismos de forma intuitiva los representaremos mediante diagramas. A estos diagramas les daremos también el nombre de grafos, aun cuando los términos y definiciones no estén limitados únicamente a los grafos que pueden representarse mediante diagramas.\\

Un grafo es un conjunto de puntos y un conjunto de líneas donde cada línea une un punto con otro. \\

A cualquier arista de un grafo se le puede asociar una pareja de vértices del mismo. Si $u$ y $v$ son dos vértices de un grafo y la arista $a$ esta asociada con este par, escribiremos $a = uv$. La representación interna del software asocia los siguientes datos a la variable arista: Aristas = $\langle$ vértice\_origen, v\_destino, coste arista $\rangle$\\

Por ejemplo, si\\
\[ V = \{v_1, v_2, v_3, v_4, v_5\} \]
y\\
\[ A = \{v_1v_2, v_1v_3, v_1v_4, v_2v_4, v_2v_5\} \]
entonces el grafo $G = (V,A)$ tiene a $v_1, v_2, v_3, v_4$ y $v_5$ como vértices y sus aristas son $v_1v_2, v_1v_3, v_1v_4, v_2v_4$ y $v_2v_5$.\\

