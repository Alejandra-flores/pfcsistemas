\chapter{Introducción}
\label{chap:introduccion}

Muchas de las aplicaciones computacionales, naturalmente, no implican sólo un conjunto de elementos o items, sino también un conjunto de conexiones de pares de dichos elementos o items. Las relaciones implícitas en estas conexiones nos plantean inmediatamente algunas preguntas: ¿Hay alguna forma de ir de un punto a otro siguiendo las conexiones realizadas? ¿Sobre cuantos elementos o items se puede navegar partiendo de uno del conjunto dado? ¿Cual es la forma más óptima de llegar desde un elemento a otro del conjunto?\\

La teoría de grafos, es una rama importante de las matemáticas combinatorias que se ha estudiado intensamente durante cientos de años. Muchas propiedades importantes y útiles de los grafos se han demostrado, sin embargo, otros muchos problemas de mucha complejidad (problemas de complejidad NP o irresolubles en tiempo polinómico) siguen sin resolverse.\\

Al igual que muchos de los dominios de otros problemas que se han estudiado, la investigación sobre teoría algorítmica de grafos es relativamente reciente. Aunque algunos de los algoritmos fundamentales son viejos, la mayoría de los más destacados han sido descubiertos en las últimas décadas. Incluso los algoritmos más simples de grafo son de utilidad en programas informáticos y los algoritmos no tan triviales son algunos de los más elegantes e interesantes jamás desarrollados.\\

Siempre será habitual el interés de saber cuál es el algoritmo más eficiente para resolver un problema que la propia resolución de este. El estudio de las características de rendimiento de los algoritmos de grafos es un reto porque:\\

\begin{itemize}
  
\item El coste de un algoritmo no sólo depende de las propiedades del conjunto de elementos o items, sino también en numerosas propiedades de las conexiones de dichos conjuntos (y las propiedades globales del grafo que se implican en dichas conexiones o uniones de elementos del conjunto).

\item Los modelos exactos de los tipos de grafos posibles que pueden aparecer son difíciles de desarrollar.

\end{itemize}

A menudo trabajaremos con los límites de coste del peor caso para los algoritmos de grafos, a pesar de que podrían representar estimaciones pesimistas sobre el desarrollo real en ciertas situaciones. Afortunadamente existen una serie de algoritmos que son óptimos e implican un pequeño uso de trabajo para su resolución. Habrá algoritmos que se podrán analizar con exactitud para situaciones específicas, pero cuando no fuera posible dicho análisis tan exacto, se tendrá que prestar especial atención a las propiedades de los distintos grafos que se podrán dar en situaciones reales o prácticas y evaluar como estas propiedades pueden afectar al rendimiento de los algoritmos.\\

Comenzaremos trabajando con las definiciones básicas y propiedades de los grafos, usando para ello una nomenclatura estándar para describir los contenidos. A continuación, se define el TAD (Tipo Abstracto de Datos) que será nuestra estructura de trabajo y forma de representación de los grafos computacionalmente independiente del lenguaje de programación o de alto nivel usado para su implementación. Las dos estructuras de datos más importantes para la representación de grafos son matriz de adyacencias o listas de adyacencias.\\

La elección entre las dos estructuras de datos depende principalmente de si el grafo de trabajo es denso o disperso, aunque como siempre, la naturaleza de las operaciones que se utilizarán juegan un papel importante en la decisión sobre cual estructura usar.\\

\section{Objetivos}

El objetivo principal del proyecto es elaborar una aplicación informática que sea capaz de representar un grafo y los posibles algoritmos de procesado que se pueden aplicar a los grafos.

La aplicación deberá cumplir los siguientes requisitos:

\begin{itemize}
\item Se mostrará en una primera instancia la representación del grafo una vez se haya introducido los valores asociados para las aristas o vértices.

\item Una vez seleccionado el algoritmo de procesamiento o la operación pertinente se mostrará el grafo origen y el resultado de la operación en la misma ventana para resaltar los cambios efectuados por el procesamiento del algoritmo.

\item Habrá un listado con los algoritmos más reseñables de la historia sobre teoría de grafos, así como una breve descripción de ellos y un ejemplo desarrollando la resolución que se dio para solventarlos. 

\end{itemize}

Es importante aclarar el dominio de conocimiento sobre los algoritmos de teoría de grafos que se han seleccionado. Los algoritmos tratados en este proyecto y su resultado en la aplicación pertenecen en su gran mayoría a temario desarrollado en las asignaturas de Matemáticas Discretas, Estructuras de Datos II y Análisis de Algoritmos I y II de la titulación de Ingeniero Técnico Informático de Sistemas de la Universidad de Cádiz.

El desarrollo del contenido teórico se realizará en un lenguaje sencillo y claro para permitir que un estudiante universitario de cualquier Ingeniería Informática puede comprender los contenidos matemáticos sin apenas dificultad añadida.

\section{Contexto: Historia sobre la teoría de grafos}

El primer artículo sobre teoría de grafos fue escrito por el famoso matemático suizo Euler, y apareció en 1736. Desde un punto de vista matemático, la teoría de grafos parecía, en sus comienzos, bastante insignificante, puesto que se ocupaba principalmente de pasatiempos y rompecabezas. Sin embargo, avances recientes en las matemáticas y, especialmente, en sus aplicaciones han impulsado en gran medida la teoría de grafos. Si ya en el siglo XIX se usaban los grafos en áreas como la teoría de circuitos eléctricos o los diagramas moleculares, hay en la actualidad parcelas de las matemáticas - la teoría de relaciones matemáticas, por ejemplo - en las que los grafos son una herramienta natural; además, han surgido muchas aplicaciones a cuestiones de carácter práctico: emparejamientos, problemas de transporte, flujo en redes y lo que, en general, se engloba bajo el nombre de "programación". La teoría de grafos ha hecho acto de presencia en campos tan dispares como la economía, la psicología o la biología, y todo ello sin renunciar a los pasatiempos, en especial si incluimos entres ellos al famoso problema de los cuatro colores, que intriga hoy a los matemáticos tanto como el primer día.\\

Dentro de las matemáticas, la teoría de grafos se considera una rama de la topología; no obstante, también está muy relacionada con el álgebra y la teoría de matrices.\\

Los grafos de intervalos aparecen al tratar una amplía variedad de problemas. \\

Arqueología. Numerosos arqueólogos han utilizado grafos de intervalos al tratar de ordenar ciertos acontecimientos de manera cronológica. En un experimento, un grupo de arqueólogos investigó los objetos encontrados en un gran número de tumbas, con la intención de ordenarlas cronológicamente. Partiendo de la base de que si dos utensilios diferentes aparecían juntos en una misma tumba, entonces sus períodos de tiempo debían solaparse, los arqueólogos construyeron un grafo en el que los vértices correspondían a los objetos, y las aristas a los pares de objetos hallados juntos en una misma tumba. Representando este grafo como un grafo de intervalos, e interpretando cada intervalo como un período de tiempo durante el cual se usó el artefacto, pudieron finalmente ordenar las tumbas cronológicamente.\\

Análisis literario. También se han usado grafos de intervalos a la hora de investigar la posible autoría de obras literarias discutidas, tales como ciertas obras de Platón. Se estudia la aparición de diversos rasgos del estilo de un autor (como el uso del ritmo) en varias obras literarias. Dibujando un grafo en el que los vértices corresponden a dichos rasgos literarios, y las aristas a los pares de rasgos que aparecen juntos en una misma obra, llegamos a una situación muy similar a la de nuestro ejemplo arqueológico. Como antes, podemos entonces investigar si el grafo resultante puede representarse como un grafo de intervalos, lo que abre la posibilidad de ordenar las obras cronológicamente. Esta forma de proceder ha permitido, en ocasiones, relacionar el estilo de la obra literaria en disputa con el estilo del autor en cuestión, determinando así la probable autoría.\\

Genética. Los grafos de intervalos aparecieron originalmente al estudiar un problema de genética más concretamente, el de determinar si la fina estructura en el interior del gen está dispuesta o no de manera lineal. Al analizar la estructura genética de un virus en concreto, el genetista Seymour Benzer consideró las mutaciones cuyos segmentos extraviados se solapan, por lo que dibujó un grafo en el que los vértices correspondían a las mutaciones, y las aristas a los pares de mutaciones cuyos segmentos perdidos se solapaban. Representando este grafo como grafo de intervalos, pudo mostrar que (para ese virus) la evidencia en favor de una disposición lineal dentro del gen era aplastante.\\

\section{Alcance}

El proyecto Suite informática de Teoría Algorítmica de Grafos dará como resultado la aplicación Graphvisualx que cumplirá con todos los objetivos y especificaciones indicados en el apartado anterior.\\

La aplicación Graphvisualx se distinguirá en varias partes: el contenido teórico de cada algoritmo a desarrollar como posible comprensión del desarrollo del mismo y el desarrollo de los algoritmos con una entrada ya sea bien en formato fichero o introducida desde la entrada estándar en el momento de ejecución de la aplicación\\

Todo lo que se encuentra desplegado en la aplicación viene descrito en notación formal matemática (algoritmos descritos bajo código fuente de la aplicación) describiendo su análisis coste temporal y algunos ejemplos al uso. (Tener en cuenta que el usuario tiene ciertas nociones básicas sobre lo que es un grafo, sus posibles representaciones y operaciones básicas de conjuntos).\\

\section{Visión global}

En cuanto a la estructura de esta Memoria del Proyecto de Fin de Carrera, tras este capítulo donde se presentan los objetivos y la visión en general del proyecto, se expone el desarrollo de las definiciones básicas de un grafo así como la mayoría de términos relacionados con él.\\

El capítulo siguiente contiene la descripción general de todos los algoritmos desarrollados para esta aplicación en una notación formal matemática y mediante algunos ejemplos de uso del algoritmo según unos valores de entrada aleatorios. \\

Además se incluirá otro capítulo en donde tendrá especial relevancia el concepto de notación asintótica y NP-Complejidad. \\

Finalmente, se presentan las conclusiones generales obtenidas una vez realizado el proyecto para pasar inmediatamente a los manuales de usuario y de instalación. \\

Además se presentan las referencias bibliográficas donde se incluyen las fuentes consultadas para la elaboración de este proyecto, un resumen que engloba las generalidades fundamentales de la aplicación, una guía de utilización (manual de usuario), una guía de instalación, el desarrollo del calendario mediante un diagrama de Gantt describiendo con detalle las distintas etapas de desarrollo del proyecto y finalmente la licencia completa del documento. \\

\section{Software utilizado}

En la realización de este proyecto se ha empleado el lenguaje de programación Java SE 1.6, empleando además la utilidad de openJDK para dicha plataforma Java.\\

Para la realización y estructuración del proyecto se ha empleado la herramienta IDE NetBeans v. 7.0. \\

La documentación y las figuras expuestas en este documento y derivados se han creado mediante el lenguaje de interpretación \LaTeX\ y para las figura el paquete de opciones de tikz.\\

Véase \ref{cap:Acceder} ``Como acceder a la suite''  si surgieran problemas.

\subsection{Licencia} 

La aplicación gráfica Graphvisualx es software libre: usted puede redistribuirlo y / o modificar bajo los términos de la Licencia Pública General de GNU según lo publicado por la Free Software Foundation, ya sea la versión 3 de la Licencia, o (a su elección) cualquier versión posterior.\\

Este programa se distribuye con la esperanza de que sea útil, pero SIN NINGUNA GARANTÍA, incluso sin la garantía implícita de COMERCIALIZACIÓN o IDONEIDAD PARA UN PROPÓSITO PARTICULAR. Ver el GNU General Public License para más detalles.\\

Debería haber recibido una copia de la Licencia Pública General de GNU junto con este programa. Si no es así, consulte \href{http://www.gnu.org/licenses/}{GNU Licenses}.\\