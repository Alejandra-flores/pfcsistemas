\chapter{Conclusiones}
\label{chap:conclusiones}

Durante la realización de la aplicación \emph{Graphvisualx: Suite informática de Teoría Algorítmica de Grafos} como trabajo de fin de carrera, he conseguido profundizar mi conocimiento sobre este mundo matemático y sus posibilidades de aplicaciones en la vida cotidiana cuya utilización va más allá de lo que cabría pensar. Además he profundizado en el conocimientos sobre la complejidad algorítmica y sus problemas importantes como son la Indecibilidad y la Intratabilidad. También he aprendido a manejar un nuevo lenguaje de programación, en este caso Java, dado que en la titulación hay muy pocas asignaturas en las cuales se imparta porque se sustituye por el lenguaje C y C++ en su mayoría. Dicho hecho ha ocasionado más de un retraso a la hora de realizar las implementaciones de los algoritmos y estructuras de datos, debido a la falta de práctica con este lenguaje como se comenta anteriormente.\\

Una vez terminado este proyecto se espera haber cumplido con los objetivos propuestos:
\begin{itemize}
\item Se mostrará en una primera instancia la representación del grafo una vez se haya introducido los valores asociados para las aristas o vértices.

\item Una vez seleccionado el algoritmo de procesamiento o la operación pertinente se mostrará el grafo origen y el resultado de la operación en la misma ventana para resaltar los cambios efectuados por el procesamiento del algoritmo.

\item Habrá un listado con los algoritmos más reseñables de la historia sobre teoría de grafos, así como una breve descripción de ellos y un ejemplo desarrollando la resolución que se dio para solventarlos. 

\end{itemize}

Ha sido un largo y tedioso trabajo durante los últimos 6 meses que será liberado de forma oficial bajo licencia libre una vez presentado el proyecto ante un tribunal para su evaluación. Este hito supondrá la culminación de esta primera etapa de proyecto, a partir de la cuál podrá ser leído y trabajado por aquel que lo considere interesante. La siguiente etapa sería la de revisión, mejora y ampliación del contenido.\\

\section{Expectativas de futuro}

El producto obtenido de este proyecto de fin de carrera es un producto final. Por su naturaleza está ideado en principio para la trata y uso de los posibles algoritmos implementados en un principio, dado que no se facilita ningún soporte directo para la implementación de un mayor contenido algorítmico debido a ausencia del mismo o recientes descubrimientos de nuevas teorías algorítmicas.\\

Se podrían implementar algoritmos de teoría de grafos sobre redes de comunicaciones o relaciones de ontología de la web que dado a la falta de conocimientos en la materia se han obviado para una posible actualización futura.\\

